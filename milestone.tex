\documentclass[a4paper]{article}
\usepackage{amsmath, amssymb, amsthm}
% \usepackage[backend=biber]{biblatex}
% \addbibresource{ref.bib}

\usepackage{syntax}
\usepackage{stmaryrd}


\title{Milestone Report\\
	\small Verifying Output Distribution Equivalence for Rectangular Discrete Probabilistic Programs via the PGF transformer semantics
}
\date{Finished on \today}
\author{Cheng Peng (2020533068)}

\DeclareMathOperator*{\PGF}{PGF}
\DeclareMathOperator*{\SOP}{SOP}
\DeclareMathOperator*{\VARS}{Var}
\DeclareMathOperator*{\PARMS}{Parm}
\DeclareMathOperator*{\iid}{iid}
\DeclareMathOperator*{\lfp}{fix}
\renewcommand{\S}[1]{ \llbracket #1 \rrbracket }


\begin{document}

\maketitle

% In the milestone, try to formulate a concrete problem, including:
% - What are the probabilistic programs you want to verify,
% - What are the properties?
% - What approaches are you going to take? Perhaps give a brief survey from the referred paper.

\begin{abstract}
	Randomized behavior is ubiquitous and inevitable in real-world programs, especially in certain security-critical domains like cryptography and cyber-physics systems.
	However, the lack of the efficient algorithms and data structures for representing and manipulating distributions makes it difficult to automatically verify the correctness of probabilistic programs.
	Transitional methods are only capable of computing digests of probability distributions, e.g., the expectation and variance, which are insufficient.\par
	Recently, the MOVES research team headed by Prof. Katoen Joost-Pieter at RWTH Aachen proposed probability generating function transformer semantics of probabilistic programs and devised automated verification techniques based the semantics.
	This novel approach preserves the entire distribution carried by programs, enabling precise reasoning about the behavior of probabilistic programs.\par
	In this course project, we will investigate into the theoretical foundation of the approach, the PGF transformer semantics, and develop proof-of-concept tools that can verify a simple program generates exactly the desired distribution.
\end{abstract}

\pagebreak

\section{Motivation}

\begin{itemize}
	\item Randomness is pervasive, arising naturally and inevitably in certain context.
	      For example, applications interacting with the physical world have to handle noisy input.
	      Models of crowd dynamics often involves non-deterministic motions.
	      Furthermore, randomness is a key to efficient algorithms and data structure.
	\item Probabilistic programs often function as the key component of security-critical systems,
	      for example, cryptograpy libraries are bedrock of online banking systems.
	      Therefore, it is necessary to verify the correctness of probabilistic programs.
	\item It is common to require that the output distribution of a program matches exactly with a pre-determined distribution.
	      Unaware of tiny deviation from the desired distribution may lead to side-channel leak or introduce vulnerabilites.
	      Previous approaches focusing on deriving expecation or establishing bounds are thus insufficient,
	      since they are not sensitive to small perturbations.
\end{itemize}

\section{Problem Statement}

\subsection{Description}
Given a probabilistic program and a distribution specification, verify that the output distribution of the program is exactly the specified one.

\subsection{Input Program}
The probabilistic program inputs will be written in ReDiP, a restricted version of pGCL.
ReDiP stands for Rectangular Discrete Probabilistic Programming Language.
As the name suggest, it random variables in ReDiP are always integer-valued,
and all branching and/or looping conditions must be rectangular, i.e., in the form \(x > n\) for a variable \(x\) and a constant \(n\).

The basic syntax of ReDiP is summarized as follows:

\begin{itemize}
	\item Empty program: \texttt{skip}
	\item Input declaration: \texttt{nat <input>}, \texttt{bool <input>}
	\item Simple statements:
	      \begin{itemize}
		      \item Assign a natural number to a variable. \texttt{<var> := <const>}
		      \item Sample from a built-in distribution. \texttt{<var> := <dist>(<parms>)}
		      \item Add a variable to another. \texttt{<var> += <var>}
		      \item Self-decrement of a variable. \texttt{<var>{-}{-}}
	      \end{itemize}
	\item Sequential composition of two ReDiP programs: \texttt{P;Q}
	\item Branch: \texttt{if(x<n)\{P\}else\{Q\}}
	\item Loop: \texttt{while(x<n)\{P\}}
\end{itemize}

Other constructions can viewed as syntax sugar built upon the basic constructions.
We can define the operational semantics of ReDiP which reflects programmers' intuition.

\subsection{Specification}
For simplicity, we use a loop-free ReDiP program as a specification. This allows specifying (1) the builtin distributions, (2) location-scale transforms of a expressible distribution, and (3) convolution of two expressible distributions.\par
We can see that the problem of verifying a program outputs a correct distribution is reduced to verifying that two ReDiP programs are equivalent.

\section{Solution Overview}

\emph{Key idea}: Formal power series algebra is more powerful than linear algebra.

\subsection{PGF transformer semantics}

Instead of encoding the input/output of a program as a higher-dimensional vector, we choose to embed them into probability generating functions (PGFs).
Say that a program contains variables \(\VARS\cup\PARMS = \{X_1,X_2,\ldots X_k\}\), then the program state can be represented by the following formal power series.
\[
	g = \sum_{\sigma\in \mathbb{N}^{k}} \Pr(\mathbf{X} = \sigma)\prod_{i=1}^k X_i^{\sigma_i}
\]

Thus a ReDiP program is a mapping that takes in a input PGF, and returns a another PGF if the program terminates.
This inspires defining the semantics of ReDiP programs as a PGF transformer.
The following table captures the semantics of basic constructions of ReDiP. Here \(\S{P}: \PGF\to\PGF\) is the semantics of \(P\) and \(g\) is the input PGF.\par

\begin{center}
	\begin{tabular}{ll}
		\hline
		syntax \(P\)                  & semantics \(\S{P}\)                                                                    \\
		\hline
		skip                          & \(id\)                                                                                 \\
		\(x := n\)                    & \(g\mapsto g[X/1]X^n\)                                                                 \\
		\(x --\)                      & \(g\mapsto (g-g[X/0])X^{-1} + g[X/0]\)                                                 \\
		\(x += \iid(D, y)\)           & \(g\mapsto g[Y/Y\S{D}[T/X]]\) where \(\S{D}\) is PGF of \(D\) with indeterminate \(T\) \\
		\(D\)                         & PGF of builtin distribution \(D\). .                                                   \\
		\(P;Q\)                       & \(\S{Q} \circ \S{P}\)                                                                  \\
		if \((x<n)\) \{P\} else \{Q\} & \(g\mapsto \S{P}(g_{x<n}) + \S{Q}(g-g_{x<n})\)                                         \\
		while \((x<n)\) \{P\}         & \(\lfp F_{x<n,P}\) where \(F_{x<n,P}(F)=g\mapsto (g-g_{x<n}) + F(\S{P}(f_{x<n})) \)    \\
		\{P\} [p] \{Q\}               & semantics of \(x = \iid(Bernoulli(1-p),1)\); if \((x<1)\) \{P\} else \{Q\} \\
		\hline
	\end{tabular}
\end{center}

To make least fixed point well-defined, we have to introduce a partial order on PGFs. (Omitted)\par
Key observation: The PGF transformer of every ReDiP program is linear.

\subsection{Handling loops with fixed point induction}

For every universally almost-surely terminating (UAST) loop, there exists a loop-free program who has equivalent semantics.
Fixed point induction is applied to reduce equivalence checking for loopy programs into equivalence checking for loop-free programs.

\subsection{Deciding loop-free program equivalence}

As we have mentioned, the PGF transforms of ReDiP programs are linear.
\emph{Recall} To verify two linear functions are equivalent, we only need to check equality on a basis of the input space.
For the PGF transformer semantics, the input space contains all valid PGFs over \(\VARS\cup\PARMS\). This formal power series ring admits a basis, the set of all point-mass PGFs.\par
Second order PGF (SOP) is introduced to compress many PGFs into one PGF with more meta indeterminate.
We use a generalized Dirac distribution SOP to encode represent the set of all point-mass PGFs.
SOP transformer semantics is developed mirroring to the PGF transformer semantics. It can be shown that SOP transformer semantics also have nice linearity.\par
Finally, we can have the following reductions which leads to a equivalence checking algorithm.
Here \(\mathcal{L}(V)\) is the set of all linear transforms whose domain and co-domain are \(V\).

\begin{enumerate}
	\item For \(f, g: \mathcal L (\PGF\S{\mathbf X})\), decide \(f\equiv g\).
	\item For \(f, g: \mathcal L (\PGF\S{\mathbf X})\), decide \(\forall x\in X . f(\delta_{x}) = g(\delta_{x})\).
	\item For \(f, g: \mathcal L (\SOP\S{\mathbf X, \mathbf U})\), decide \(f\equiv g\).
	\item For \(f, g: \mathcal L (\SOP\S{\mathbf X, \mathbf U})\), decide \(f(\delta_{u,x}) = g(\delta_{u,x})\).
\end{enumerate}

So we can have the following procedure for deciding equivalence of ReDiP two programs \(P\) and \(Q\), where \(Q\) is loop-free.

\begin{enumerate}
	\item Construct \(\delta_{u,x}=\prod_{i=1}^k {(1-X_i U_i)}^{-1}\)
	\item Compute \(f=\S{P}(\delta)\) recursively.
	\item Compute \(g=\S{Q}(\delta)\) recursively.
	\item Check whether \(f=g\) holds.
\end{enumerate}

For UAST ReDiP programs, their SOP transformer semantics preserves closed form, so we don't have to handle infinite formal power series.
All the computations can be done in a computer algebra system like \texttt{sympy}.\\

\emph{One last thing: verifying UAST using techniques from previous related research works.}

\appendix
\section{Progress Report}

\begin{center}
	\begin{tabular}{ll}
		\hline
		task                                                                                  & progress          \\
		\hline
		Comprehend the CAV'22 PRODIGY paper                                                   & (done)            \\
		Pick up basic usage of the \texttt{probably} package, which contains a parser of pGCL & (done)            \\
		Implement a algorithm for recursively evaluate the SOP transformer semantics          & (not started yet) \\
		Implement the main algorithm for equivalence checking                                 & (not started yet) \\
		Experiments                                                                           & (not started yet) \\
		Presentation                                                                          & (just started)    \\
		Final report                                                                          & (just started)    \\
		\hline
	\end{tabular}
\end{center}

% \appendix
% \setlength{\parskip}{0pt}
% \printbibliography

\end{document}
