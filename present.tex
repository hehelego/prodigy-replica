\documentclass[11pt]{beamer}

\usetheme{Madrid}
\usepackage{multicol}
\usepackage{biblatex}
\usepackage{ulem}
\addbibresource{ref.bib}

\usepackage{tikz}
\usepackage{graphicx}
\graphicspath{{figures/}}
\DeclareGraphicsExtensions{.png,PNG,.jpg,.JPG,.jpeg,.JPEG,.pdf,.PDF}
\usepackage{subcaption}
\usepackage{amsmath, amsthm, amssymb}

\DeclareMathOperator*{\argmax}{arg\,max}
\DeclareMathOperator*{\argmin}{arg\,min}

\title{PGF transformer semantics}
\author{Cheng Peng \and Dantong Liu}
\date{\today}

\AtBeginSection[] % Do nothing for \section*
{
	\begin{frame}<beamer>
		\frametitle{Contents in this section}
		\begin{multicols}{2}
		\tableofcontents[currentsection]
		\end{multicols}
	\end{frame}
}

% 10 minutes presenting
% 5 minutes Q&A
\begin{document}

\maketitle
\begin{frame}{You may have doubts}
	The theme: \emph{PGF transform semantics} based \emph{equivalence checking} of pGCL programs

	\hfill

	\begin{block}{Equivalence checking?}
		This work is about linearity no normalization is ever involved.
	\end{block}

	\begin{block}{I heard the problem is undecidable!}
		We will focus on a restricted fragment of pGCL called ReDiP.
	\end{block}
\end{frame}

\section{Introduction}
\subsection{Backgrounds}
\begin{frame}{Backgrounds}

\end{frame}
\subsection{Limitations of previous works}
\begin{frame}{Motivation}

\end{frame}

\section{Formulation}
\subsection{Research questions}
\begin{frame}{The research questions}

\end{frame}
\subsection{Overview of the solution}
\begin{frame}{Overview of our solution}

\end{frame}

\section{pGCL and ReDiP}
\subsection{Syntax and Semantics of pGCL}
\begin{frame}{Syntax of pGCL}

\end{frame}
\begin{frame}{Semantics of pGCL}

\end{frame}
\subsection{Restrictions on pGCL: ReDiP}
\begin{frame}{A fragement of pGCL: ReDiP}

\end{frame}

\section{PGF transformer semantics}
\subsection{Review on PGFs}
\begin{frame}{Review on PGFs}

\end{frame}
\subsection{PGF transformer semantics of ReDiP}
\begin{frame}{PGF transformer semantics}

\end{frame}
\subsection{Handling loops with fixed point induction}
\begin{frame}{Fixed Point Induction}

\end{frame}
\subsection{Properties of the PGF transformer semantics}
\begin{frame}{Linearity \& Closed-form preservation}

\end{frame}
\begin{frame}{Summary}

\end{frame}

\section{Evaluation}
\subsection{Setup}
\begin{frame}{Benchmark setup}

\end{frame}
\subsection{Results}
\begin{frame}{Results}

\end{frame}
\subsection{Case Study}
\begin{frame}{Case Study}

\end{frame}

\section{Discussion}
\subsection{Conclusion of this project}
\begin{frame}{Conclusion}
\end{frame}
\subsection{Future works}
\begin{frame}{Limitations and future works}
\end{frame}
\begin{frame}[c]{Q \& A}
	Questions are welcomed.
\end{frame}

\appendix

\section{references}
\begin{frame}[allowframebreaks]{references}
	\nocite{*}
	\renewcommand*{\bibfont}{\tiny}
	\printbibliography
\end{frame}

\end{document}
